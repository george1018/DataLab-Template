\documentclass[aspectratio=169, t]{beamer}
% \geometry{paperwidth=170.67mm, paperheight=96mm}
% \usepackage{enumitem}% http://ctan.org/pkg/enumitem
\usepackage{lmodern}
\usepackage{etoolbox}
\usepackage{ragged2e}
\usepackage{caption}
\usepackage{subcaption}
\usepackage{listings}
\usepackage{verbatim}
\usetheme{Madrid}
\usecolortheme{beaver}


\definecolor{scolor}{RGB}{178,34,34}
\definecolor{autumnred}{RGB}{219, 83, 83}
\setbeamerfont{section number projected}{%
  family=\rmfamily,series=\bfseries,size=\normalsize}
\setbeamercolor{section number projected}{bg=white,fg=scolor}

\setbeamertemplate{section in toc}[ball]

\setbeamercolor{item}{fg=scolor}

\setbeamercolor{block title example}{use=example text,fg=white,bg=autumnred}
\setbeamercolor{block body example}{parent=normal text,use=block title example,bg=block title example.bg!10!bg}



\newcommand\itemshape[1]{%
  \setbeamertemplate{itemize item}[#1]%
  \usebeamertemplate{itemize item}%
}

\setbeamertemplate{enumerate items}[default]
\setbeamertemplate{theorems}[numbered] 


\setbeamercolor{block title definition}{fg=blue,bg=blue!20!bg}
\setbeamercolor{block body definition}{bg=block title.bg!30!bg}


\BeforeBeginEnvironment{definition}{%
    \setbeamercolor{block title}{fg=white,bg=red!70!black}
    \setbeamercolor{block body}{fg=black, bg=red!30!white}
}
\AfterEndEnvironment{definition}{
 \setbeamercolor{block title}{use=structure,fg=structure.fg,bg=structure.fg!20!bg}
 \setbeamercolor{block body}{parent=normal text,use=block title,bg=block title.bg!50!bg, fg=black}
}


\apptocmd{\frame}{}{\justifying}{} % Allow optional arguments after frame.
\setbeamertemplate{caption}[numbered]


%Information to be included in the title page:
\title[Workshop on AI Meets Econometrics]{Workshop on AI Meets Econometrics}
\subtitle{From Machine Learning to Deep Learning}
\author{Fei Wang}
\institute[Goethe University Frankfurt]{Management and Microeconomics \\
 Goethe Universität Frankfurt}
\date{25 May 2022}



\begin{document}
\frame[noframenumbering]{\titlepage}

\begin{frame}{Roadmap}
    \tableofcontents
\end{frame}

\section{Introduction}


\begin{frame}{}
    \vspace{90pt}
    \begin{center}
        \huge INTRODUCTION 
    \end{center}
\end{frame}

\begin{frame}{Introduction}
    \begin{columns}
        \begin{column}{0.45\textwidth}
            The Big Picture
        \end{column}
        \begin{column}{0.45\textwidth}
            \vspace{5mm} \\ 
            How will we demystify machine learning and deep learning?
            \begin{enumerate}
                \item Review of BLUE 
                \item Review of Causal Inference
                \item From Econometrics to Machine Learning
                \item From Machine Learning to Deep Learning
                \item A Concrete Example: NLP
            \end{enumerate}
        \end{column}
    \end{columns}
    \footnotesize{
            \url{https://en.wikipedia.org/wiki/Deep_learning}
            }
\end{frame}


\section{Review of BLUE}


\begin{frame}{}
    \vspace{90pt}
    \begin{center}
        \huge REVIEW OF BLUE
    \end{center}
\end{frame}

\begin{frame}{Review of BLUE}
    For the following multivariate linear regression model,
    \begin{equation*}
        \begin{bmatrix}
            y_1 \\
            y_2 \\
            \vdots \\
            y_t \\
            \vdots  \\ 
            y_N
        \end{bmatrix} = \begin{bmatrix}
            1 & x_{12} & \cdots & x_{1k} \\
            1 & x_{22} & \cdots & x_{2k} \\
            \vdots & \vdots & \vdots \\
            1 & x_{t2} & \cdots & x_{tk} \\ 
            \vdots & \vdots & \vdots \\
            1 & x_{N2} & \cdots & x_{Nk} 
        \end{bmatrix} \begin{bmatrix}
            \beta_1 \\ 
            \beta_2 \\
            \vdots \\
            \beta_t \\ 
            \vdots \\
            \beta_k
        \end{bmatrix} + \begin{bmatrix}
            u_1 \\
            u_2 \\
            \vdots \\
            u_t \\
            \vdots  \\ 
            u_N
        \end{bmatrix}
    \end{equation*} 
    or simply:
    \begin{equation}
        Y = X \beta + u 
    \end{equation}
    where, $Y$ is a $N \times 1$ vector, $X$ is a $N \times k$ matrix, and 
    $\beta$ is a $k \times 1$ vector. 
    With assumptions:
    \begin{align*}
        & E(u)  = 0 \ \ \ \ 
        V(u)  = \sigma^2 I \\ 
        & \text{X is fixed} \ \ \ \ 
        \text{X  has full rank k (or $k < N$)} 
    \end{align*}
\end{frame}

\begin{frame}{Review of BLUE: Assumptions Matter}
    The model is built up for \textbf{population}:
    \begin{equation*}
        Y = X \beta + u 
    \end{equation*}
    For each \textbf{sample}, we use OLS method to estimate $\hat{\beta_1}, \cdots, \hat{\beta_k}$
    that minimize the sum of \textbf{squared residuals}:
    \begin{equation*}
        S = \sum_{i=1}^N \hat{u_t}^2 = (Y-X\hat{\beta})'(Y-X\hat{\beta})
    \end{equation*}
    and yields the following estimator in matrix form:
    \begin{equation}
        \hat{\beta} = (X'X)^{-1}X'Y 
    \end{equation}
\end{frame}

\begin{frame}{Review of BLUE: Assumptions Matter (1)}
    OLS estimator $\hat{\beta}$ is the Best \textbf{Linear Unbiased} Estimator
    because we assume $E(u) = 0$. With the following population model and estimator:
    \begin{align*}
        Y & = X\beta + u \\ 
        \hat{\beta} & = (X'X)^{-1}X'Y = (X'X)^{-1} X'(X\beta + u) \\ 
        \hat{\beta} & = \beta + u \\ 
        \to E(\hat{\beta})       & = E(\beta + u) = E(\beta) + E(u)  \\
        E(\hat{\beta})  & = \beta 
    \end{align*}
    The Gauss-Markov theorem also states that the variance of OLS estimator $\hat{\beta}$
    has the \textbf{smallest variance}:
    \begin{equation*}
        V(\hat{\beta}) = \sigma^2(X'X)^{-1},
    \end{equation*}
    which make it become the \textbf{Best Linear Unbiased Estimator} (BLUE).
\end{frame}

\begin{frame}{Review of BLUE: Assumptions Matter (2)}
    We reviewed one assumption
    The variance of $\hat{\beta}$ is:
    \begin{equation*}
        V(\hat{\beta}) = \sigma^2(X'X)^{-1},
    \end{equation*}
    However, $\sigma^{2}$ is unknown as it is assumed for the \textbf{population}
    model. We have to \textbf{estimate} $\sigma^2$. By imposing the \textbf{second assumption}:
    \begin{equation*}
        V(u) = \sigma^2 I
    \end{equation*}
    we could have the following equation:
    \begin{equation*}
        E\Big( \frac{\hat{u}'\hat{u}}{N-k} \Big ) = \sigma^2 
    \end{equation*}
    Therefore, an \textbf{unbiased} estimator of $\sigma^2$ is provided by
    \begin{equation*}
        \hat{\sigma}^2 = \frac{\hat{u}'\hat{u}}{N-k} 
    \end{equation*}
\end{frame}

\begin{frame}{Review of BLUE: Assumptions Matter (3)}
    Now, we check the third assumption: $X$ is fixed (or non-stochastic), which
    could be interpreted as `distribution is fixed' rather than stochastic (for
    example, a random walk). With this assumption, we could safely add one more
    assumption:
    \begin{equation*}
        u \sim N(0, \sigma^2)
    \end{equation*}
    The above assumption \textbf{was not} included in the model that assumes:
    \begin{enumerate}
        \item $E(u) = 0$
        \item $V(u) = \sigma^2 I$
        \item X is fixed 
        \item X has full rank (k < N)
    \end{enumerate}
    When $X$ is fixed and error term $u$ (for population model) follows the normal distribution
    $$u \sim N(0, \sigma^2)$$, we could have: 
    \begin{equation*}
        \hat{\beta} \sim N(\beta, \sigma^2(X'X)^{-1})
    \end{equation*}
\end{frame}

\begin{frame}{Review of BLUE: Assumptions Matter (3)}
    One more step! Now, we have the following distribution:
    \begin{equation*}
        \hat{\beta} \sim N(\beta, \sigma^2(X'X)^{-1})
    \end{equation*}
    However, we don't know $\sigma^2$. After replacing $\sigma^2$ with the
    unbiased estimator $\hat{\sigma}^2$, it turned out that the hypothesis such as
    $\beta_i = \beta_{i0}$ could be tested by comparing the test statistic:
    \begin{equation*}
        \frac{\hat{\beta}_i - \beta_{i0}}{\sqrt{\hat{\sigma}^2\alpha_{ii}}}
    \end{equation*}
    with \textbf{critical values} from the $t_{N-k}$ distribution, where $\alpha_{ii}$
    is the $ii$'s element of $(X'X)^{-1}$. 
\end{frame}

\begin{frame}{Review of BLUE: Assumptions Matter (4)}
    The final assumption: $X$ is full rank ($k < N$)
    Try to solve the following different cases:
    \begin{itemize}
        \item $\beta_1 + 2\beta_2 = 3$, infinite solutions (k = 2, N = 1)
        \item unique solutions ($k = 2, N = 2$, full rank) 
        \begin{align*}
            & \beta_1 + 2 \beta_2 = 3 \\
            & \beta_1 - \beta_2 = 5
        \end{align*}
        \item unique solutions ($k < N$) too 
        \begin{equation*}
            \begin{bmatrix}
                y_1 \\
                y_2 \\
                \vdots \\
                y_N
            \end{bmatrix} = \begin{bmatrix}
                1 & x_{12} & \cdots & x_{1k} \\
                1 & x_{22} & \cdots & x_{2k} \\
                \vdots & \vdots & \vdots \\
                1 & x_{N2} & \cdots & x_{Nk} 
            \end{bmatrix} \begin{bmatrix}
                \beta_1 \\ 
                \beta_2 \\
                \vdots \\
                \beta_k
            \end{bmatrix} + \begin{bmatrix}
                u_1 \\
                u_2 \\
                \vdots  \\ 
                u_N
            \end{bmatrix}
        \end{equation*}
        \begin{equation*}
            N \times 1 \ \ \ \ \ \ \ \ \ \ \ \ \ \ N \times k \ \ \ \  \ \ \ \ \ \ k \times 1 \ \ \ \ N \times 1 
        \end{equation*}
    \end{itemize}
\end{frame}

\begin{frame}{Review of BLUE: Summary}
    \begin{itemize}
        \item $E(u) = 0$ gives us Linear Unbiased Estimator
        \item $V(u) = \sigma^2 I$ gives us BLUE
        \item $V(u) = \sigma^2 I$ gives us $\hat{\sigma}^2$ too
        \item X is fixed and $u \sim N(0, \sigma^2)$ enable us to do t-test
        \item X is full rank makes sure that we could solve equations  
    \end{itemize}
    Why bother to do all those things? 
    \begin{itemize}
        \item collecting data is not easy (sample size is often small)
        \item making inference on population needs the assumption of distribution 
        \item having a framework of conducting hypothesis test
        \item confidence in assumed distribution gives the confidence in inference
        \begin{itemize}
            \item it is often the case
            \item Law of Large Numbers (LLN) and Central Limit Theorem (CLT) guarantees this 
        \end{itemize}
    \end{itemize}
\end{frame}

\end{document}